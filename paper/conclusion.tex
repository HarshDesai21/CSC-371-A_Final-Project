
\section{Conclusions}
\label{sec:concl}

% In this section, briefly summarize your paper --- what problem did you
% start out to study, and what did you find? What is the key result /
% take-away message? It's also traditional to suggest one or two avenues
% for further work, but this is optional.

Overall, our results demonstrated that the gender of MOBA video game characters could be predicted with moderate accuracy, with a higher accuracy for male characters compared to female characters. Our best model, the SVM with the RBF kernel trained on the standardized dataset, was 63\% accurate with a weighted F1 score of 0.64. The results across the initial and tuned models emphasize that using non-linear decision boundaries improved classification for the minority class, and had the best balance of results. Some models had higher overall accuracy, but that was due to their prioritization of classifying the majority class, while suffering in accuracy on female characters. Across all models, it appears as if it was a difficult classification task for the minority gender especially (Women), which might also be a result of less representation in MOBA games and our dataset. However, if we had fully accurate results, that would imply that the models were able to clearly discriminate between male and female characters, which would highlight how character stats are possibly dependent on their gender. Because the predictions were moderate, it does not appear that design bias in character design led to significant results that improved classification. However, this paper has a novel attempt for a gender analysis of video game characters, using a Machine Learning approach to predict character gender. It is difficult to necessarily interpret these results, then, as supporting or refuting our hypothesis that accurate results would point towards gender biases by developers in creating characters.


\subsection{Ethical Considerations and Broader Impacts}

This work deals with gender biases and embedded inequalities in video game development. As discussed in the background, the video game industry and community is stereotypically a masculine space, where women and minorities may have experienced discrimination or harassment. This can carry over to developers as well, where character designs and stats may be different depending on the gender of the character. This work aims to highlight these biases, and not further contribute to them, offering an approach to analyze gender differences in video game characters by attempting to distinguish and predict them. The impacts of our models are a bit complicated, as strong results would suggest clear gender differences (high bias/inequality), while weaker results suggest that the classification is more difficult (less bias, characters are designed not according to gender). We believe our results are somewhere in the middle - our overall accuracy was moderate, suggesting that gender was not extremely easy to predict, but our models were much less accurate in correctly predicting female characters. One interpretation of this is that male characters may be designed with a certain archetype or are more similar to one another, having consistent stats such as higher HP or defense, while female characters may be more 'random' with less similarities. However, this would have to be investigated further, and it would be ideal to have a more balanced dataset to test theories further.

\subsection{Future Work}

One limitation of this work was in the size of our dataset. In total, our dataset had 640 characters. Future work could consider increasing the size of the dataset by including other games. However, one difficulty of this project was deciding how to overlap game statistics, since mechanisms work differently depending on the game. For this reason, the dataset had less features than it could have, in order to focus on features that could be somewhat cleanly collapsed on each other. This also meant that all the games included had to stay within-genre. It would be interesting to use a similar approach to analyze character differences in different video game genres, such as Role Playing Games (RPGs) or Hero Shooters. However, MOBAs were also selected due to their large roster sizes, so this analysis on another genre might be even less feasible due to there being less characters in other games. 

Additionally, there are several other approaches towards analyzing gender differences based on our dataset. Using statistical testing, such as t-tests, correlation tests, or another analyses can point to how different stats might be higher or lower depending on gender. This can be advantageous towards indicating specific features that are more subject to bias from developers, using test scores and \textit{p} values to identify where there are significant differences. Because of the scope and goal of this project, we did not employ these measures, but it may be useful for future work. 